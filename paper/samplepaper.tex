\documentclass[runningheads]{llncs}
%
\usepackage{graphicx}
\usepackage{proof} 
\usepackage{amsfonts, mathtools}
\usepackage{listings}
\usepackage[linkcolor=blue]{hyperref}
\urlstyle{same}
\newcommand{\tN}{\mathbb{N}}

\begin{document}
%
\title{A Sound Strategy to Compile General Recursion into Finite Depth Pattern Matching}
%
\titlerunning{Compiling Recursion into Finite Pattern Matching}
% If the paper title is too long for the running head, you can set
% an abbreviated paper title here
%
\author{Maycon J. J. Amaro\inst{1}\orcidID{0000-0001-7684-8428}\and
Samuel S. Feitosa\inst{2}\orcidID{0000-0002-9485-4845}\and
Rodrigo G. Ribeiro\inst{1}\orcidID{0000-0003-0131-5154}}
%
\authorrunning{Amaro, Ribeiro and Feitosa}
% First names are abbreviated in the running head.
% If there are more than two authors, 'et al.' is used.
%
\institute{Universidade Federal de Ouro Preto, Ouro Preto --- MG, Brazil\and
Universidade Federal da Fronteira Sul, Chapecó --- SC, Brazil
}

\maketitle              % typeset the header of the contribution
%
\begin{abstract}
  %this one has 128 words. Organization defines 15--250.
  Programming languages are popular and diverse, and the convenience
  of programmatically changing the behavior of complex systems is attractive 
  even for the ones with stringent security requirements, which often 
  impose restrictions on 
  the acceptable programs. A very common restriction is that the 
  program must terminate, which is very hard to check because the 
  Halting Problem is undecidable. In this work, we proposed a 
  technique to unroll recursive programs in functional languages to 
  create terminating versions of them. We prove that our strategy 
  itself is guaranteed to terminate. We also formalize term generation
  and run property-based tests 
  to build confidence that the semantics is 
  preserved through the transformation.
  Our strategy can be used to compile general purpose functional 
  languages to restrictive targets such as the eBPF and smart 
  contracts for blockchain networks.

\keywords{program transformation  \and recursion \and program generation}
\end{abstract}

\section{Introduction}

Looping statements and recursion are one of the 
most common features of programming languages,
but they also require a lot of caution. Non-termination 
is at best
an annoying situation, and at worst a serious 
security or logical concern. Some 
compilers and technologies put a lot effort 
in guaranteeing that no program will run forever,
inevitably being very restricted due 
to the undecidability of the Halting 
Problem~\cite{sipser1996introduction}. Some 
examples include dependently typed languages 
that are used as proof assistants, such as 
Coq~\cite{huet1997coq} and Agda~\cite{norell07}; smart contract languages for 
blockchain systems~\cite{le2018} and the technology to run 
sandboxed programs in the Linux 
Kernel---the eBPF (extended Berkeley Packet Filter).

eBPF~\cite{ebpf22} is an interesting case to explore, 
because its verifier will reject any program 
that has a back jump. In other words, it will 
reject any form of 
repetition\footnote{Version 5.3 and higher 
has support for 
bounded loops only}, be it iterative or 
recursive.
Their motivation in doing so is 
understandable: running programs in the 
Operating System's kernel requires a lot of 
caution. One mistake could bring the system
down.
Allowing potential non-terminating programs 
is a breach that ill-intentioned users could explore to 
perform Denial of Service attacks.

While imperative languages have additional 
constructs for repetition, functional 
languages can only 
count with recursion for repeating 
computations. Bounded loops are one easy way 
to walk around the restriction that all 
programs 
must terminate, but using functional 
languages 
to write programs targeting eBPF requires 
better strategies. Notice that ensuring termination 
alone is not 
sufficient. A program with repetition, even with a 
proof that it terminates, would still be rejected
by eBPF. Termination must be syntactically assured.

In this work, we present an algorithm to transform 
recursive functions into finite depth pattern 
matching functions with an {\it equivalent} 
semantics, in the sense that both functions must 
yield the same results when the recursive function 
halts and the non-recursive function, with the given input, has 
enough nested pattern matching constructions to produce a value. 
We also present a sound strategy to generate random terminating programs
so our algorithm can be properly tested.
More specifically, we make the following contributions:  

\begin{itemize}
\item We present System R, a core language with 
recursion and its unrolling algorithm.
\item We present System L, a core language with no 
recursion and show how to compile system R programs 
into equivalent System L programs.
\item We describe a sound algorithm to generate random 
well-typed terminating programs for the System R 
language.
\item We formally demonstrate some properties regarding 
the presented algorithms. Property-based tests are 
applied otherwise.
\end{itemize}

\newcommand{\stlc}{$\lambda_\rightarrow$}

\section{Basic Definitions}
We consider the simply typed $\lambda$-calculus, 
frequently represented as \stlc~\cite{pierce2002}, 
extended with natural numbers, pattern matching 
over naturals and recursion, as defined in~\cite{plfa20.07}. 
Its syntax is described by the following context-free grammar:

\begin{center}

\begin{tabular}{r l}
  $\tau$ ::=& $\tN$ \textbar \, $\tau \rightarrow \tau$\\
  $e$ ::=& zero \textbar \, suc $e$ \textbar \, $v$ \textbar \, $\lambda v:\tau.e$ \textbar \, $e \: e$
  \textbar \, match $e$ $e$ ($v$, $e$) \textbar \, $\mu v:\tau.e$ 
\end{tabular}
\end{center}

\noindent We use the metavariable $v$ to range over 
variable names, 
$\tau$ to range over types and $e$ to range 
over 
expressions. We use {\it expressions} 
and {\it terms} interchangeably. \stlc's type 
system is described by the following set of 
rules:

\[\arraycolsep=2em
\begin{array}{r r}
\infer[_{\{znat\}}]{\Gamma \vdash \text{zero} : \tN}{} &
\infer[_{\{snat\}}]{\Gamma \vdash \text{suc }e:\tN }{\Gamma \vdash e : \tN}\\ \\
\infer[_{\{lam\}}]{\Gamma \vdash (\lambda v:\tau.e) : \tau \rightarrow \tau'}{\Gamma , v : \tau \vdash e : \tau'} &
\infer[_{\{var\}}]{\Gamma \vdash v : \tau}{v : \tau \in \Gamma}\\ \\ 
\infer[_{\{app\}}]{\Gamma \vdash e \: e' \: : \tau'}{\Gamma \vdash e : \tau \rightarrow \tau' & \Gamma \vdash e' : \tau} &
\infer[_{\{rec\}}]{\Gamma \vdash (\mu v:\tau.e) : \tau}{\Gamma , v : \tau \vdash e : \tau}
\end{array}
\]
\[
\begin{array}{c}
\infer[_{\{match\}}]{\Gamma \vdash \text{match } e_1 \: e_2 \: (v, e_3) : \tau}{\Gamma \vdash e_1 : \tN & \Gamma \vdash e_2 : \tau & \Gamma , v : \tN \vdash e_3 : \tau}
\end{array}
\]

\noindent Rule $znat$ types zero as a natural number in any 
context. Rule $snat$ types suc as a natural number 
as long as its argument is also a natural number.
Rule $var$ types any element of the context with 
its typing information. Rules $lam$ and $app$ 
type abstractions and applications. Rule 
$match$ requires its 
first argument to be a natural number and the 
others to have the same type. The third argument 
expression is bound to a variable name intended to 
refer to the predecessor of the first argument, if 
it is not zero. Finally, rule $rec$ types 
recursive expressions (fixpoint operator).

The semantic style considered in this work is the small step, {\it call by value} semantics following~\cite{plfa20.07}. In the text, capture-avoiding substitution of variable $x$ by term $y$ in term $e$ is represented by $[x \mapsto y]e$.

% \[
% \begin{array}{c c}
% \infer[_{\{suc\}}]{\text{suc }e \longrightarrow \text{suc }e'}{e \longrightarrow e'} &
% \infer[_{\{app_1\}}]{v_1 \: e_2 \longrightarrow v_1 \: e_2}{e_2 \longrightarrow e_2'} \\\\
% \infer[_{\{app_3\}}]{e_1 \: e_2 \longrightarrow e_1' \: e_2}{e_1 \longrightarrow e_1'} & 
% \infer[_{\{rec\}}]{\mu v : \tau. e \longrightarrow [v \mapsto \mu v : \tau . e]e}{}
% \end{array}
% \]
% \[
% \begin{array}{c}
% \infer[_{\{app_2\}}]{(\lambda v : \tau . e) \: v_1 \longrightarrow [v \mapsto v_1]e}{}
% \\\\
% \infer[_{\{match_1\}}]{\text{match }e_1 \: e_2 \: (v, e_3) \longrightarrow \text{match }e_1' \: e_2 \: (v, e_3)}{e_1 \longrightarrow e_1'} \\\\
% \infer[_{\{match_2\}}]{\text{match zero } e_2 \: (v, e_3) \longrightarrow e_2}{} \\\\
% \infer[_{\{match_3\}}]{\text{match (suc $e$) } e_2 \: (v, e_3) \longrightarrow [v \mapsto e]e_3}{}
% \end{array}
% \]

% Rule $\{suc\}$ evaluates the subterm to produce the
% successor of the value, $\{app_1\}$, $\{app_2\}$ 
% and $\{app_3\}$ represents the call by value strategy, 
% evaluating both terms to a value before applying.
% Rule $\{rec\}$ evaluates recursive functions, 
% $\{match_1\}$, $\{match_2\}$ and $\{match_3\}$ model the 
% match construction evaluating to the second term if the 
% first is zero and to the third term if the first is the 
% successor of a number. The third term might use the 
% predecessor of the first term, available through the 
% variable $v$.

\section{Expansion and Transformation}
Transformation of recursive functions into 
terminating versions is done considering the two core 
calculi we defined: System R, which is a subset of 
the presented $\lambda$-calculus also featuring 
some form of recursion; and the strongly 
normalizing System L. Translating programs from 
System R to System L implies recursion 
elimination. In order to still be able to perform non 
trivial computations, System R's expressions are 
expanded before 
the translation, unrolling the recursive 
definitions up to a factor we called \textit{fuel}
\footnote{The term ``fuel'' is inspired by Petrol 
Semantics. It is presented, for instance, in ~\cite{mcbride2015turing}.}.
In this way, the output functions still computes 
something identical to the original function, 
although they can never have the same exact 
semantics, since one is recursive and the other has no 
automated repetition at all.

System R's syntax establishes two levels for expressions, 
reserving the top level for recursive definitions 
and applications only. The bottom level is 
constituted of the same constructs seen in 
$\lambda$-calculus, except for the $\mu$ operator.
We made this choice to prevent nested fixpoint 
expressions. For example, we want to avoid expressions 
of the form $\mu v_1 : \tau . (\mu v_2 : \tau . e)$.
In functional languages such as Haskell and Racket, the 
programmer  
is able to define several functions before defining a 
main expression. Those definitions can be compiled into 
derived forms, abstracting the main expression and 
applying it to the function definitions. The semantics 
of those applications will create nested $\mu$ functions. 
As a syntatic transformation, our 
unrolling is intended to happen before semantic 
evaluation. Our choice to allow recursive functions 
only at top level is just a matter of presentation: 
we can allow several 
definitions to be unrolled separately with some minor 
tweaks 
(by using a {\tt let in} construction and several fuel values, for example). Once semantic evaluation nests 
them, they will be already expanded. With some extra 
effort, the core ideas in this paper should apply to 
the traditional \stlc as well, although it is a conjecture 
at this point.

Recursive definitions also require a functional 
type, so trivial loops of type $\tN$ cannot be 
expressed (for example, the term $\mu v : \tN . v$).
The following context-free grammar 
describes System R's syntax:

\begin{center}
\begin{tabular}{r l}
  $\tau$ ::=& $\tN$ \textbar \, $\tau \rightarrow \tau$\\
  $p$ ::=& $\mu v : \tau\rightarrow\tau.e$ \textbar \: $p \: e$ \\
  $e$ ::=& zero \textbar \, suc $e$ \textbar \, $v$ \textbar \, $\lambda v:\tau.e$ \textbar \, $e \: e$
  \textbar \, match $e$ $e$ ($v$, $e$)
\end{tabular}
\end{center}

\noindent Meta-variable $p$ ranges over top-level 
expressions. For convenience, we assume every program will 
contain a recursive definition, regardless if it is involved 
in a application or not. Typing rules remain the same as 
\stlc \hspace*{1pt} except for 
$rec$ rule that now requires the functional type.
The new $rec$ rule is:

\[
\begin{array}{c}
\infer[_{\{rec\}}]{\Gamma \vdash (\mu v:\tau\rightarrow\tau'.e) : \tau}{\Gamma , v : \tau\rightarrow\tau' \vdash e : \tau\rightarrow\tau'}
\end{array}
\]

\noindent Alternatively, the typing rule for recursive functions in 
System R could require the expression to have an ocurrence of the variable 
that was added to the context. This is convenient for proving 
theorems but is burdensome for random program generation and 
real life implementations. Variable ocurrence are 
inductively defined following 
Definition~\ref{def:call}. 

\begin{definition}[Variable ocurrence]\label{def:call}
A variable $v : \tau$ occurs in some term $e$, 
denoted by $v : \tau \in_{v} e$, if $e$ is $v$ or 
if $v : \tau$ occurs in any of the subterms of 
$e$. Otherwise, $v : \tau$ does not occur in $e$, 
denoted by $v : \tau \not \in_v e$.
\end{definition}

\noindent Informally, the semantics of System R is the same of \stlc. 
But since System R's syntax won't allow a fixpoint inside 
a bottom-level term, the conventional evaluation rule 
for recursive functions
cannot be applied. The solution is to elaborate System R
to \stlc, which is straightforward, given System R programs 
are a subset of \stlc. In this way, only during evaluation 
recursive definitions are allowed to bypass the syntactic 
restrictions. Notice that there is 
no need to modify the rule to reflect the 
constraint of fixpoints having functional type, because 
that is just a particular case of the more general rule 
already present in \stlc.

\subsection{Unrolling}
The inlining of a function $f$ means to replace 
the ocurrences of $f$ for its body, and is largely used 
by developers of compilers to increase performance 
of their programs~\cite{appel04}. By 
inlining recursive definitions into themselves, 
we create an equivalent expression that performs 
fewer recursive calls. As Rugina and 
Rinard~\cite{rugina00} point out, we need to be 
careful to not super-exponentially grow the code 
of the functions. For example, consider this 
pseudo-Haskell function for adding two numbers in Peano notation:

\begin{lstlisting}[language=Haskell, mathescape=true]
sum :: Nat $\rightarrow$ Nat $\rightarrow$ Nat
sum = \x y $\rightarrow$ case x of 
  Zero  $\rightarrow$ y
  Suc w $\rightarrow$ sum w (Suc y)
\end{lstlisting}

% $\mu s : \tN \rightarrow \tN \rightarrow \tN .\lambda x : \tN . \lambda y : \tN .\text{match } x \: y \: (w, (s \: w) \: (\text{suc } y))$.

\noindent The idea in this expression is to match the first 
number $x$ over the patterns zero or suc. If it is 
zero, we simply answer with the second number $y$. 
Otherwise we name $w$ the predecessor of $x$ and 
recursively call the function over $w$ and the 
successor of $y$. Inlining this function with itself 
once would give us the following expression:

\begin{lstlisting}[language=Haskell, mathescape=true]
sum :: Nat $\rightarrow$ Nat $\rightarrow$ Nat
sum = \x y $\rightarrow$ case x of 
  Zero  $\rightarrow$ y
  Suc w $\rightarrow$ (\x y $\rightarrow$ case x of 
    Zero  $\rightarrow$ y
    Suc w $\rightarrow$ sum w (Suc y)) w (Suc y)
\end{lstlisting}

% \begin{gather*}
% \mu s : \tN \rightarrow \tN \rightarrow \tN .\lambda x : \tN . \lambda y : \tN . \text{match } x \: y \: (w, ((\lambda x : \tN . \lambda y : \tN.\\ \quad \text{match } x \: y \: (w, (s \: w) \: (\text{suc } y)) \: w) \: (\text{suc } y)) 
% \end{gather*}
\noindent We had one match ({\tt case of}, in Haskell) before and now we have two. If we 
inline this last expression with itself we would end up 
with four matches, because the definition of $sum$ now contains 
two of them:
\begin{lstlisting}[language=Haskell, mathescape=true]
sum :: Nat $\rightarrow$ Nat $\rightarrow$ Nat
sum = \x y $\rightarrow$ case x of 
 Zero  $\rightarrow$ y
 Suc w $\rightarrow$ (\x y $\rightarrow$ case x of 
  Zero  $\rightarrow$ y
  Suc w $\rightarrow$ (\x y $\rightarrow$ case x of 
   Zero  $\rightarrow$ y
   Suc w $\rightarrow$ (\x y $\rightarrow$ case x of 
    Zero  $\rightarrow$ y
    Suc w $\rightarrow$ sum w (Suc y)) w (Suc y)) w (Suc y)) w (Suc y)
\end{lstlisting}
% \begin{gather*}
%   \mu s : \tN \rightarrow \tN \rightarrow \tN .\lambda x : \tN . \lambda y : \tN . \text{match } x \: y \: (w, ((\lambda x : \tN . \lambda y : \tN . \\\text{match } x \: y \: (w, ((\lambda x : \tN . \lambda y : \tN . \text{match } x \: y \: (w, ((\lambda x : \tN . \lambda y : \tN . \\\text{match } x \: y \: (w, (s \: w) \: (\text{suc } y))) \: w) \: (\text{suc } y))) \: w) \: (\text{suc } y))) \: w) \: (\text{suc } y)) 
%   \end{gather*}

\noindent Doubling the number of 
matchings every step in a simple term like this 
is not in our best interest.
Instead, we want to increase these matches just by 
one in each step. Inlining the original expression 2 
times should actually give us the following term:

\begin{lstlisting}[language=Haskell, mathescape=true]
sum :: Nat $\rightarrow$ Nat $\rightarrow$ Nat
sum = \x y $\rightarrow$ case x of 
  Zero  $\rightarrow$ y
  Suc w $\rightarrow$ (\x y $\rightarrow$ case x of 
    Zero  $\rightarrow$ y
    Suc w $\rightarrow$ (\x y $\rightarrow$ case x of 
      Zero  $\rightarrow$ y
      Suc w $\rightarrow$ sum w (Suc y)) w (Suc y)) w (Suc y)
  \end{lstlisting}
% \begin{gather*}
%   \mu s : \tN \rightarrow \tN \rightarrow \tN .\lambda x : \tN . \lambda y : \tN . \\\text{match } x \: y \: (w, ((\lambda x : \tN . \lambda y : \tN . \\\text{match } x \: y \: (w, ((\lambda x : \tN . \lambda y : \tN . \\\text{match } x \: y \: (w, (s \: w) \: (\text{suc } y))) \: w) \: (\text{suc } y))) \: w) \: (\text{suc } y)) 
% \end{gather*}

\noindent If the original function
had two recursive calls, then we would want to increase 
the number of matches by two in each step, and that 
would still be better than having $2^{m+1}$ matches
after $m$ steps. 
So, we have to always keep track 
of the original expression and inline its body 
inside the output expressions. 
Definition~\ref{def:inl} formalizes this behavior.

\begin{definition}\label{def:inl}
The $n$-expansion of a System R expression $e$ is the 
result of the $n$th cumulative inlining of $e$, 
described by $exp$ in the following algorithm:
\begin{align*}
inl(v, v', e) =& \begin{cases}
  e & \text{if } v \equiv v'\\
  v & \text{otherwise}
\end{cases}\\
inl(\text{zero}, v', e) =& \, \text{zero}\\
inl(\text{suc }e', v', e) =& \, \text{suc }(inl(e', v', e))\\
inl(e_1 \: e_2, v', e) =& \, (inl(e_1, v', e)) \: (inl(e_2, v', e))\\
inl(\lambda v : \tau.e', v', e) =& \, \lambda v : \tau.(inl(e', v', e))\\
inl(\text{match }e_1 \: e_2 \: (w, e_3), v', e) =& \, \text{match }(inl(e_1, v', e)) \: (inl(e_2, v', e)) \\ & (w, (inl(e_3, v', e)))
\end{align*}
\begin{align*}
exp'(e', \_, \_, 0) =& \, e'\\
exp'(e', v, e, n) =& \, inl(exp'(e', v, e, n-1), v, e)
\end{align*}
\begin{align*}
exp(\mu v : \tau\rightarrow\tau'.e, n) =& \, \mu v : \tau\rightarrow\tau' . (exp'(e, v, e, n))\\
exp(p\:e, n) =& \, (exp(p, n)) \: e  
\end{align*}
\end{definition}

\noindent Function $inl(e', v, e)$ performs the inlining of 
$e$ inside $e'$ replacing the ocurrences of $v$. 
This is only allowed between bottom-level 
expressions, so the fixpoint construction can never 
appear here. Function $exp'(e', v, e, n)$ 
accumulates the inlinings $n$ times, always using 
the original $e$ instead of previous result with 
itself. Function $exp(e, n)$ expands a recursive 
function $v$ by $n$-expanding its body replacing 
the ocurrences of $v$. Functions $inl$, $exp'$ and $exp$ 
halt for every input (Lemmas~\ref{lem:inl} and~\ref{lem:exp}) and the resulting term is 
still a recursive function (Lemma~\ref{lem:pres}), 
whose recursive calls 
must be now eliminated. 

\begin{lemma}[Inlining Halts]\label{lem:inl}
For all bottom-level terms $e, e'$ of System R,
and every variable $v$, there exists a bottom-level term 
$e''$ such that ${inl(e', v, e)=e''}$.\end{lemma}
\begin{proof}
By induction on the structure of $e$. Base cases for 
variable and zero immediately produce a term. In the 
other cases inlining is merely propagated to subterms, 
which are strictly smaller.  \qed
\end{proof}  

\begin{lemma}[Expansion Halts]\label{lem:exp}
For every $n \in \tN$ and System R expression $e$, 
there exists an expression $e'$ such that 
$exp(e) = e'$.
\end{lemma}
\begin{proof}
The expansion of bottom-level terms, done in the $exp'$ 
function, always halts because inlining halts and $n$ is 
always decreasing towards $0$, in which case the 
expression is left unchanged. Top-level expressions come 
in two cases. 
\begin{itemize}
  \item Case 1: $e = \mu v : \tau \rightarrow \tau'.e_b$. Its 
  body $e_b$, which is a bottom-level expression, is 
  expanded.
  \item Case 2: $e = (p) \: (e_r)$. The top-level $p$ is expanded 
  with the same $n$, bottom-level $e_r$ is left unchanged. 
  By the structure of the grammar, a top level expression will 
  always end with a recursive definition, falling in case 1.
\end{itemize} 
 \qed
\end{proof}

\begin{lemma}[Occurrence Preserving Expansion]\label{lem:pres}
For every $n \in \tN$, term $e$ and variable $v : \tau$, 
if $v : \tau \in_v e$ then $v : \tau \in_v exp(e, n)$.
\end{lemma}
\begin{proof}
Inlining substitutes all ocurrences of a variable $v_1$ by 
some 
$e_1$ inside some $e_2$. If $v_1$ occurs in $e_1$ and in 
$e_2$, then by 
induction, $v_1$ occurs in $inl(e_2, v_1, e_1)$.
The expansion of a recursive definition 
$\mu v : \tau \rightarrow \tau'.e$ inlines $e$ within 
$e$ replacing 
$v$ in every step. So if $v$ occurs in $e$ then it follows 
that $v$ still occurs after each cumulative inlining, 
and thus after the expansion. \qed
\end{proof}

\subsection{Recursion Elimination}
System L's syntax offers no way to express any 
kind of recursion, but offers a construction for 
abnormal termination, namely the term {\tt error}:

\begin{center}
\begin{tabular}{r l}
  $\tau$ ::=& $\tN$ \textbar \, $\tau \rightarrow \tau$ \\
  $e$ ::=& zero \textbar \, suc $e$ \textbar \, $v$ \textbar \, $\lambda v:\tau.e$ \textbar \, $e \: e$
  \textbar match $e$ $e$ ($v$, $e$) \textbar \: error 
\end{tabular}
\end{center}

\noindent Its type system is identical to \stlc \hspace*{1pt} minus the 
rule for recursion, and adding the rule for typing 
error with an arbitrary type:

\[
\begin{array}{c}
\infer[_{\{error\}}]{\Gamma \vdash \text{error } : \tau}{\forall \tau}
\end{array}
\]

\noindent The semantics for System L 
is the same as \stlc's but including three rules 
for propagating errors and removing the $\{rec\}$ 
rule. The error construction, although a 
normal form, is not considered a value. In this 
way, the semantics remains 
deterministic~\cite{pierce2002}.
The included rules are:
\[\arraycolsep=2em
\begin{array}{c c}
\infer[_{\{esuc\}}]{\text{suc error} \longrightarrow \text{error}}{} &
\infer[_{\{eapp_1\}}]{\text{error} \: e_2 \longrightarrow \text{error}}{}\\\\
\infer[_{\{eapp_2\}}]{v_1 \: \text{error} \longrightarrow \text{error}}{} &
\infer[_{\{ematch\}}]{\text{match error }e_1 \: e_2 \: (v, e_3) \longrightarrow \text{error}}{}
\end{array}
\]

\noindent Since System L has no way of expressing recursion or any
form of repetition there is no way a program in System L 
loops forever. Translating a term from System R to System L will 
produce a program that will surely terminate. To achieve this, we must 
eliminate the remaining recursive calls, i.e., the occurrences of the function 
name in the expression.
Definitions~\ref{def:rec} and~\ref{def:rec2} formally describe recursion 
elimination and translation.

\begin{definition}\label{def:rec}
Ocurrence elimination of a variable $v$ from a bottom-level term $e$ is the translation of $e$ to System L, replacing $v$ ocurrences by error. It is defined by this algorithm:
\begin{align*}
elim(v', v) =& \begin{cases}
  \text{error} & \text{if }v \equiv v'\\
  v' & \text{otherwise}
\end{cases}\\
elim(\text{zero}, v) =& \, \text{zero}\\
elim(\text{suc }e, v) =& \, \text{suc }(elim(e, v))\\
elim(\lambda v' : \tau . e, v) =& \, \lambda v' : \tau . (elim(e, v))\\
elim(e_1 \: e_2, v) =& \, (elim(e_1, v)) \: (elim(e_2, v))\\
elim(\text{match }e_1 \: e_2 \: (v', e_3), v) =& \, \text{match }(elim(e_1, v)) \: (elim(e_2, v)) \\ & (v', elim(e_3, v))
\end{align*}
\end{definition}

\begin{definition}\label{def:rec2}
The translation of a System R expression $e$ is the ocurrence  
elimination of the recursive function's name from its own body:
\begin{align*}
tsl(\mu v : \tau \rightarrow \tau' . e) =& \, elim(e, v)\\
tsl(p \: e) =& \, (tsl(p)) \: e
\end{align*}
\end{definition}

\noindent The second case of $tsl$ leaves the right term of 
application unchanged, since bottom-level System R 
expressions are a subset of System L. In an actual 
implementation, an auxiliary function is necessary to 
simply translate bottom-level terms into their System 
L's counterparts. The composition of 
unrolling and translation gives us the transformation algorithm, which always halts (Theorem~\ref{theo:hal}):
\begin{align*}
\mathit{transform}(e, f) = tsl \, (exp(e, f))
\end{align*}

\begin{lemma}[Ocurrence Elimination Halts]\label{lem:hal}
For every bottom-level System R term
$e$ and variable $v$, there exists a System L 
expression $e'$ such that ${elim(e, v)=e'}$.\end{lemma}
\begin{proof}
Ocurrence elimination behaves similar to 
inlining. In inlining, the ocurrences of $v$ are replaced by 
some term $e_1$. In occurence elimination, the ocurrences of $v$ are replaced 
by the term {\it error}.
By induction,
if the term is a variable or a zero, it is 
immediately translated by error or its 
counterpart in System L and then halts.
If not, the ocurrence elimination is propagated to the 
subterms, which are strictly smaller. \qed
\end{proof}

\begin{theorem}[Transformation Halts]\label{theo:hal}
For every $f \in \tN$ and System R expression $e$, there exists a System L expression $e'$ such that 
${\mathit{transform}(e, f) = e'}$.\end{theorem}
\begin{proof}
Translation erases the $\mu$ construction, 
which is absent in System L, and then 
eliminates the ocurrences of the recursive function's name from the body.
Transformation composes $n$-expansion and 
translation. Since both functions always 
halt, their composition always halts too. \qed
\end{proof} 

\noindent In the example program for adding two numbers 
$x$ and $y$, when the ocurrences of the function's name are eliminated in 
the translating process, the resulting expression can 
still answer with $y$ if $x$ is $0$:
$\mu s : \tN \rightarrow \tN \rightarrow \tN . \lambda x : \tN . \lambda y : \tN . \text{match } x \: y \: (w, (\text{error } w) \: (\text{suc } y))
$.

If we expand the expression $f$ times before 
translating, the System L version would be 
able to successfully compute the sum whenever 
$x \leq f$, reducing to error otherwise. 
There can be no general way of guessing $f$, 
since it could solve the halting problem. One 
disadvantage of our approach is to rely on 
the programmer to inform the fuel for the 
function, similar to what they have to do 
when using bounded loops in imperative 
languages. One advantage of this approach is 
that some infinite loops are immediately 
reduced to error. No matter how high is $f$, 
if we transform $\mu v : \tN \rightarrow \tN . v$ 
the output program will always be one simple 
error.

\section{Term Generation}
Property-based testing is an interesting approach to testing, 
because it relies on generating random programs to check 
properties, avoiding the bias when developers create 
manual test cases. But using this approach when checking 
properties of compilers is not easy, because it is necessary 
to define how to generate valid programs of that language.
In this work, since we are directly dealing with termination, 
we are also interested in generating terminating programs.

To accomplish this, we define for our generation procedure a 
superset of types, in which our base type $\tN$ is indexed by 
natural numbers. Those indexes are related via subtyping, 
such that $\tN^x <: \tN^y$ whenever $x \leq y$. Functions 
are related in the usual way, being covariant in return type 
and contravariant in the argument type. Our generation 
judgment $\Gamma; d; r; \tau \leadsto e$ means that 
expression $e$ of type $\tau$ can be generated from context 
$\Gamma$, given limits $d$ for expression depth and $r$ for 
type indexes. The relation $\leadsto$ is annotated with a 
letter to distinguish the form of the expressions it 
generates. $\xi(xs)$ selects a random element from a non-empty list $xs$. Our first rules are the generation of 
zero and terms that are variables. A zero can be generated in every 
scenario where a $\tN$ is expected, regardless of the index.
Variables can be picked from a non empty list of candidates. 
Those candidates are the variables from the context that are 
a subtype of the expected type.
\[\arraycolsep=1.5em
\begin{array}{c c}
\infer[_{\{zero\}}]{\Gamma; d; r; \tN^x \leadsto_z zero}{} & 
\infer[_{\{var\}}]{\Gamma; d; r; \tau \leadsto_v \xi(cs)}{cs = \{ v \, | \, v : \tau' \in \Gamma \land \tau' <: \tau\} & cs \neq \emptyset}
\end{array}
\]

\noindent The successor construction can be also generated in every 
scenario where a $\tN^x$ is expected, as long as index $x$ is 
not already 0. Its argument can be zero, a variable or 
another successor construction. The index limit is decreased 
for the generation of its subterm. When we generate pattern matchings 
inside recursive definitions, the first branch must be a constant, 
so it is useful to not include the other rules as possible subterms 
for this construction. The 
notation $\psi, \phi, \dots = \xi(\{a, b, c, d, \dots\})$ means 
that rules annotated with $\psi$, $\phi$, etc., are an 
alias to rules annotated with any letter in the list 
inside $\xi$, and are meant to be selected randomly for each 
letter on the left side.
\[
\begin{array}{c}
\infer[_{\{suc\}}]{\Gamma; d; r+1; \tN^{x+1} \leadsto_s suc \: e}{\Gamma; d; r; \tN^x \leadsto_{\psi} e & \psi = \xi(\{z, v, s\})}
\end{array}  
\]

\noindent Abstractions need to generate an expression of the return 
type before inserting a $\lambda$. The function \emph{fresh} 
generates a variable name that is not present in the given context.
Operator $\lfloor \tau \rfloor$ erases index from indexed 
type $\tau$.
\[
\begin{array}{c}
\infer[_{\{abs\}}]{\Gamma; d; r; \tau_1 \rightarrow \tau_2 \leadsto_a \lambda v : \lfloor \tau_1 \rfloor . e}{v = \mathit{fresh}(\Gamma) & \Gamma , v : \tau_1; d; r; \tau_2 \leadsto_{\psi} e & \psi = \xi(\{z, v, s, a\})}
\end{array}
\]

\noindent Applications of type $\tau$ need the generation of a 
function that results in that $\tau$, applied to an 
appropriate argument, which has also to be generated.
Operator $\Theta$ generates a type given the limits $d$ and 
$r$, and never associates to the left. This operator is defined by randomly 
selecting one out of two more specific versions: 
$\Theta^\to$ that generates only function types
and $\Theta^\tN$ that generates only indexed $\tN$ types.

\begin{align*}
  \Theta^\tN(r) &= \tN^{\xi(\{1, \dots, r\})}\\
  \Theta^\rightarrow(0, r) &= (\Theta^\tN(r)) \rightarrow (\Theta^\tN(r))\\
  \Theta^\rightarrow(2d, r) &= (\Theta^\tN(r)) \rightarrow (\xi(\{\Theta^\tN(r), \Theta^\rightarrow(d, r)\}))\\
  \Theta(d, r) &= \xi(\{\Theta^\tN(r), \Theta^\rightarrow(d, r)\})
  \end{align*}

  \noindent If limits are high enough, both base and functional types can be generated. Otherwise, it will force the generation of a base type ($\tN$ is our only base type here). Generating 
applications will cut the $d$ limit to half
for generating its subterms.
\[
\begin{array}{c}
\infer[_{\{app\}}]{\Gamma; 2d; r; \tau \leadsto_a e_1 \: e_2}{\tau' = \Theta(d, r) & \Gamma; d; r; \tau' \rightarrow \tau \leadsto_{\phi} e_1 & \Gamma; d; r; \tau' \leadsto_{\psi} e_2 & \phi, \psi = \xi(\{z, v, s, a\})}
\end{array}
\]



\noindent Match constructions need to generate a term of type $\tN$ and 
two terms of the expected type. For the third term $e_3$, 
a fresh variable is added to context with a decreased index 
limit, and this limit is decreased for $e_3$ as well. The 
operator $_\downarrow$ decreases the index of the type. If it 
is a function, only the leftmost atom is decreased. 
Matches can only be introduced when limit $d$ is greater than 1.
\[
\begin{array}{c}
\infer[_{\{match\}}]{\Gamma; 2d; r; \tau \leadsto_a match \: e_1 \: e_2 \: (v, e_3)}{v = fresh(\Gamma) & \Gamma; d; r; \tN^r \leadsto_{\phi} e_1 &  \\ \phi, \psi, \rho = \xi(\{z, v, s, a\}) & \Gamma; d; r; \tau \leadsto_{\psi} e_2 & \Gamma, v : \tN^{r_\downarrow}; d; r_\downarrow; \tau \leadsto_{\rho} e_3 }
\end{array}
\]

\noindent The generation of recursive functions and applications 
involving them require more caution. Generating a recursive 
function needs the generation of a proper body, which means 
it has to be well-typed and must terminate. 
\[
\begin{array}{c}
\infer[_{\{rec\}}]{\Gamma; 2d; r+1; \tau_1 \rightarrow \tau_2 \leadsto_f rec \: v : \lfloor \tau_1 \rightarrow \tau_2 \rfloor . e}{v = \mathit{fresh}(\Gamma) & \Gamma , v : (\tau_1 \rightarrow \tau_2)_\downarrow; d; r_\downarrow; \tau_1 \rightarrow \tau_2 \leadsto_b e}
\end{array}
\]

\noindent The simplest way 
of guaranteeing this is inserting $\lambda$s until we are 
left with the generation of a $\tN$, and then generate a 
match. This match will generate an immediate natural number on its 
first branch and it will allow a recursive call on the second. The 
expression being matched needs to be zero or a variable.
The expression being generated for the second branch needs 
to be standardized, i.e., modified to make sure that, if 
there is a recursive call, then its first argument will be 
the predecessor of the matching expression.
\[
\begin{array}{c}
\infer[_{\{buildBody1\}}]{\Gamma; d; r; \tN^x \leadsto_b match \: e_1 \: e_2 \: (v, e_3)}{v = \mathit{fresh}(\Gamma) \\ \rho = \xi(\{z, v, s, a\}) & \Gamma; d; r; (\tN^x)_\downarrow \leadsto_{\psi} e_1 & \Gamma, v : (\tN^x)\downarrow; d; r; \tN^x \leadsto_{\rho} e_3' \\ \phi,\psi = \xi(\{z, v, s\})& \Gamma; d; r; \tN^x \leadsto_{\phi} e_2 & e_3 = \mathit{stdz}(e_3'; \Gamma , v : (\tN^x)_\downarrow)}\\\\
\infer[_{\{buildBody2\}}]{\Gamma; d; r; \tau_1 \rightarrow \tau_2 \leadsto_b \lambda v : \lfloor \tau_1 \rfloor . e}{v = \mathit{fresh}(\Gamma) & \Gamma, v : \tau_1; d; r; \tau_2 \leadsto_b e}
\end{array}
\]

\noindent The standardization process is defined by the following algorithm, where bottom is a function that returns the first 
variable name 
added to the context, i.e., the bottom of the scope stack; 
and top is a function that returns the last variable name 
added to 
the context, i.e., the top of the scope stack. 
\begin{align*}
  std(e_1 \: e_2, v_1, v_2) &= 
  \begin{cases} 
      e_1 \: v_2 & e_1 \equiv v_1\\ 
      (std(e_1, v_1, v_2)) \: (std(e_2, v_1, v_2)) & \mathit{otherwise} 
  \end{cases}\\
  std(\lambda v:\tau.e, v_1, v_2) &= \lambda v:\tau.(std(e, v_1, v_2))\\
  std(match \: e_1 \: e_2 \: (v, e_3), v_1, v_2) &= match \: (std(e_1, v_1, v_2)) \: (std(e_2, v_1, v_2)) \\&\quad (v, std(e_3, v_1, v_2))\\
  std(e, \_, \_) &= e
\end{align*}
\begin{align*}
  stdz(e, \Gamma) &= std(e, \mathit{bottom}(\Gamma), \mathit{top}(\Gamma))
\end{align*}

\noindent Applications involving recursive definitions are most useful 
if they yield a number as a final value, and so we force this 
to happen. After generating a functional type and a recursive 
definition of this type, we 
generate proper arguments and apply them to the function.
\[
\begin{array}{c}
\infer[_{\{apprec\}}]{\Gamma; 2d; r; \tN^x \leadsto_g e}{\tau = \Theta^\rightarrow(d, r) & \Gamma; d; r; \tau \leadsto_f e' & \Gamma; d; r; e'; \mathit{revargs}(\tau) \leadsto_c e}
\end{array}
\]

\begin{align*}
\mathit{args}(\tN^x) &= []\\
\mathit{args}(\tN^x \rightarrow \tau_2) &= \tN^x :: (\mathit{args} (\tau_2))\\
\mathit{revargs} &= \mathit{reverse} \circ \mathit{args} 
\end{align*}

\noindent For this, we build a list of arguments and 
extend the judgement to contain them. We must build the 
applications of the outer arguments first, applying the recursive 
definition to the first argument as the innermost application. For this, 
we need to assume our list of arguments is reversed.
\[\arraycolsep=1em
\begin{array}{c c}
\infer[_{\{bdA1\}}]{\Gamma; d; r; e; [\tN^x] \leadsto_c e \: e'}{\psi = \xi(\{z, v, s, a\}) \\ \Gamma; d; r; \tN^x \leadsto_{\psi} e'} &
\infer[_{\{bdA2\}}]{\Gamma; d; r; e; \tN^x :: ts \leadsto_c e_1 \: e_2}{& \psi = \xi(\{z, v, s, a\})\\ \Gamma; d; r; e; ts \leadsto_c e_1 & \Gamma; d; r; \tN^x \leadsto_{\psi} e_2}
\end{array}
\]

\noindent Finally, generating a System R expression means to generate a 
type and generate a term accordingly:
\[
\begin{array}{c c}
\infer[]{\Gamma; d; r; \tN^x \leadsto e}{\Gamma; d; r; \tN^x \leadsto_{\psi} e} &
\infer[]{\Gamma; d; r; \tau_1 \rightarrow \tau_2 \leadsto e}{\Gamma; d; r; \tau_1 \rightarrow \tau_2 \leadsto_{\phi} e}

\end{array}
\]

\noindent If we want only top terms of System R, then it is enough to 
use $\psi = g$ and $\phi = f$. But it is interesting to allow 
the generation of bottom level terms when testing properties, 
since recursive functions are optional in real-life 
implementations. In that case, $\psi = \xi(\{z, v, s, a, g\})$ and $\phi = \xi(\{a, f\})$.

\subsection{Soundness of Term Generation}

\begin{lemma}[Bottom level generation is sound]
  For every context $\Gamma$, natural numbers $d$ and $r$, type $\tau$ and ${\psi \in \{a, s, z, v\}}$, if ${\Gamma; d; r; \tau \leadsto_\psi e}$ then $e : \tau$. 
\end{lemma}
\begin{proof}
  The rule for \{zero\} can only be used to generate terms of type $\tN$, which is the type of zero and the rule for \{var\} pick a value of type $\tau'$ from the context if available, where $\tau'$ and $\tau$ are equal up to erasure of indices. Rules \{suc\}, 
  \{abs\} and \{match\} directly represent their counterparts in the type system, generating subterms of the appropriate type. The rule for \{app\} generates a random type to construct a functional type with return of type $\tau$ and then generates both the function and its argument, building an application with them. \qed
\end{proof}

\begin{lemma}[Bottom level generation halts]
  For every context $\Gamma$, natural numbers $d \geq 2$ and $r$, and type $\tau$, there exists a term $e$ such that ${\Gamma; d; r; \tau \leadsto_a e}$.
\end{lemma}
\begin{proof}
  Type $\tau$ is either $\tN$ or a function type. The first case has rules \{match\} and \{app\} available (since $d \geq 2$) and the other is generated by \{abs\}. When generating subterms for \{match\}, \{abs\} or \{app\}, the rules \{zero\}, \{var\} and \{suc\} can be used. Rules \{zero\} and \{var\} trivially halt. Rule \{suc\} decreases $r$ and the index of $\tN$ by 1 for the generation of its subterms and cannot be used when those values are 0. The subterm generated by rule \{abs\} has a type strictly smaller then $\tau$ and it cannot be used once $\tau$ becomes $\tN$. Rules \{app\} and \{match\} decrease the $d$ limit by half for the generation of their subterms, and cannot be used when $d < 2$. Therefore, some value in the triple $(d, r, \tau)$ will always decrease for the generation of a subterm. The rule \{zero\} can be used to generate a trivial term whenever rules \{app\}, \{match\} and \{abs\} cannot be used anymore and there are no candidates for using \{var\}. \qed
\end{proof}

\begin{theorem}[Top level generation is sound]
  For every context $\Gamma$, natural numbers $d$ and $r$, type $\tau$ and ${\psi \in \{f, g\}}$, if ${\Gamma; d; r; \tau \leadsto_\psi e}$ then $e : \tau$.
\end{theorem}
\begin{proof}
If $\tau$ is a functional type, $e$ is generated using \{rec\}.
This rule creates a fresh var of type $\tau$ and adds it to the 
context when generating a subterm of type $\tau$, satisfying the 
corresponding typing rule. The subterm is generated using a special 
procedure that builds an adequate body for the recursive function. 
This procedure generates well-typed $\lambda$ abstractions until 
$\tau$ is a single $\tN$, generating a well-typed pattern matching right away.
If $\tau$ is already $\tN$, a functional type and a recursive 
function $p$ of this type are generated by rule \{apprec\}, 
then the type of each 
argument is used to generate bottom-level values 
and build nested applications 
until $p$ has all arguments necessary to generate a $\tN$. \qed
\end{proof}

\begin{theorem}[Top level generation halts]
For every context $\Gamma$, natural numbers $d \geq 2$ and $r$, and type $\tau$, there exists a term $e$ such that either ${\Gamma; d; r; \tau \leadsto_f e}$ or ${\Gamma; d; r; \tau \leadsto_g e}$.
\end{theorem}
\begin{proof}
If $\tau$ is a functional type, it must follow that 
${\Gamma; d; r; \tau \leadsto_f e}$. 
Rule \{rec\} depends on $\leadsto_b$ for generating its subterm. By 
using \{buildBody2\}, $\tau$ is decreasing for the generation of 
the subterms of the $\lambda$ inserted abstractions. When it 
reaches $\tN$, bottom level terms are generated for the insertion 
of the match. If $\tau$ is already $\tN$, bottom level terms are 
generated for each argument type of the generated functional type, 
finishing with the application of 
the recursive function when the list of arguments contains only the 
last one. \qed
\end{proof}

\section{Quick-checking Properties}
We implemented Ringell~\cite{ringell}, an 
interpreter for both System R and System L. When given only a
source-code file, it will parse the text, typecheck the abstract 
syntax tree, 
use System R as the 
internal representation and run it using the \stlc interpreter 
as seen in~\cite{plfa20.07}. 
When given a source-code file and a  
nonnegative integer number $n$, it will transform 
the System R representation into System L, using $n$ as 
fuel for expansion, and then run it using the \stlc interpreter 
as seen in~\cite{pierce2002}. Ringell's syntax, inspired in 
Haskell and \stlc, is simple and support just a few syntactic 
constructions, making it close to both System R and System L syntax.

The term generation procedure is implemented in Ringell's test modules. 
We specified some properties using Quick 
Check~\cite{sullivan08}, invoking modules from 
Ringell for validation. The properties are as follows:
\begin{itemize}
  \item Property 1: All generated System R programs are well typed
  \item Property 2: All generated System R programs terminate
  \item Property 3: For every generated System R term $e$ of 
  type $\tN$, if $e$ terminates with value $v$ doing at most 
  $f$ recursive calls, then $\mathit{transform}(e, f)$ yields $v$. 
  \item Property 4: For every generated System R term $e$ of 
  type $\tN$ and some fuel $f$, if $\mathit{transform}(e, f)$ yields 
  a value $v$, then $e$ terminates resulting in $v$.
\end{itemize}

\noindent Property 2 is only true because of our generation 
procedure, it is obviously not a property of 
System R. Properties 3 and 4 are our desired 
semantic properties for the transformation 
technique: the output function results in the same 
value of the original if enough fuel is given and 
the output function yielding a value implies the 
original function terminates with the same value.
Both properties concern only programs of the base 
type $\tN$, because there are two syntatic 
constructors for programs of functional type in 
System R and only one in System L, and we prefer 
using propositional equality instead of creating a 
more complex equivalence relation for this purpose.
Figure~\ref{fig:coverage} presents the coverage results generated 
by Haskell Stack coverage flag, after running 
thousands of successful tests for each 
property.

\begin{figure}
  \includegraphics*[width=\textwidth]{coverage.png}
  \caption{Test Coverage Results}
  \label{fig:coverage}
\end{figure}

\noindent The first column are for modules names, 
in which {\tt ExpR} and {\tt ExpL} are the 
interpreters and {\tt Unroll} is where the 
functions involved in transformation are. Most of 
non-reached code concerns error control and 
arguments of functions that were not used in 
all cases of the function (for example, context is 
not needed for evaluating a {\tt zero}).

\section{Related Work}
There are some papers closely related to ensuring 
termination of functional programs, although the systems they 
describe do not focus on syntactically transforming the programs. 

Abel's 
foetus~\cite{abel98,abel99,abel99a,abel02} is a 
simplification of Munich Type Theory Implementation, and 
features a termination checker for simple functional 
programs. It builds an hypergraph of function calls, 
computes its transitive closure and tries to find a 
decreasing lexicographic order in the recursive 
functions' arguments. Foetus is pretty limited, forcing 
the programmer to construct several auxiliary, often 
mutually recursive, functions to convince the checker.

Barthe's Type based termination~\cite{barthe04,gilletut08,barthe08} 
is a strategy in which the language's types carries more 
information so the type system can serve as a proof 
system that the programs terminate. It is an elegant 
but complicated approach, and since it does not 
transform the programs syntactically, targets with stringent security 
requirements would still not accept the functions.

Also, LLVM's Clang~\cite{clang} offers a macro to unroll 
bounded loops in languages of C family, which is largely 
used when using pseudo-C code to compile for eBPF. 
GCC~\cite{gcc} has a command line option to unroll loops 
in C programs. The features in both compilers are unable 
to deal with recursion. At the time of this paper, 
Microsoft's MSVC~\cite{msvc} has no way to directly 
instruct the compiler to unroll a loop.

Finally, our generation procedure and test approach are 
based on the work in~\cite{feitosa2020type}. It formalizes a 
type-directed algorithm to generate random programs of 
Featherweight Java, which is used to verify several 
properties using QuickCheck.

\section{Conclusion}
It is quite useful to allow the dynamic change of behavior in
computing systems. This level of adaptation is essential to 
complete several tasks with efficiency, being attractive 
even for systems with stringent security requirements, such as the Linux kernel. 
Because infinite loops can bring a whole 
system down, a very common restriction in those
systems is that the 
programs must terminate, which is impossible to check in general. 
Although imperative languages have bounded loops to walk around 
this limitation, functional languages are left empty-handed.
In this work, we proposed a 
technique to unroll recursive programs in functional languages. For 
this task, we defined two core languages: one with recursion (System R) 
and one without recursion but with syntax for errors (System L). 
We defined an expansion algorithm for System R and a translation 
algorithm from System R to System L. Our strategy is guaranteed to 
terminate, and can be used to create 
compilers from general purpose functional languages to restricted 
scenarios such as eBPF or smart contracts for blockchain networks.
Our prototype implementation of the strategy, Ringell, 
is available to public access.

\bibliographystyle{splncs04}
\bibliography{references}

\end{document}
